%!TEX program = xelatex


\RequirePackage[safe]{tipa}

\documentclass[a4paper, zihao=-4, linespread=1]{ctexrep}
\renewcommand{\CTEXthechapter}{\thechapter}
% 最小行间间距设定
\setlength{\lineskiplimit}{3pt}
\setlength{\lineskip}{3pt}

% 中文支持
\setCJKmainfont[BoldFont=Source Han Serif SC Bold]{Source Han Serif SC}
\xeCJKsetup{CJKmath=true}
\setCJKmathfont{KaiTi}  % 数学环境中使用楷体
\newCJKfontfamily[zhxinwei]\xinwei{STXinwei} % 定义新字体

% 颜色
\usepackage[table]{xcolor}
\newcommand{\scol}[1]{\colorbox{#1}{\rule{0em}{1ex}}\,#1}

% 首字下沉
\usepackage{lettrine}

% 分栏
\usepackage{multicol}
\setlength{\columnsep}{20pt}
\setlength{\columnseprule}{0.4pt}

% 数学环境
\usepackage{mathdots} % 数学省略号,会重定义 dddot 和 ddddot
\usepackage{amsmath}  
\newcommand{\ue}{\mathrm{e}}
\newcommand{\ud}{\mathop{}\negthinspace\mathrm{d}} % 微分号
\usepackage{amssymb}
\usepackage{mathrsfs} % 线性代数字体
% overline的替代命令
\newcommand{\closure}[2][3]{{}\mkern#1mu\overline{\mkern-#1mu#2}}
\usepackage{yhmath} % \wideparen命令:弧AB
\usepackage{mathtools} % dcases环境,prescript命令
\usepackage{amsthm} % 定理环境
\theoremstyle{definition}\newtheorem{laws}{Law}[section]
\theoremstyle{plain}\newtheorem{ju}[laws]{Jury}
\theoremstyle{remark}\newtheorem*{marg}{Margaret}
\usepackage{esint} % 多重积分,需放在amsmath后
% 箭头与长等号
\usepackage{extarrows}
% 中括号的类二项式命令
\newcommand{\Bfrac}[2]{\genfrac{[}{]}{0pt}{}{#1}{#2}}

% 下划线宏包
\usepackage[normalem]{ulem}
% LaTeX符号宏包
\usepackage{hologo}
\newcommand{\xelatex}{\Hologo{XeLaTeX}}
\newcommand{\bibtex}{\Hologo{BibTeX}}
\newcommand{\biber}{\Hologo{biber}}
\newcommand{\tikzz}{Ti\textit{k}Z}
\newcommand{\bz}{B\'ezier}
% 其他符号
\usepackage{wasysym}
% 带箱小页
\usepackage{boxedminipage}
% 绘图
\usepackage{tikz}
\usetikzlibrary{calc,intersections,positioning,angles,quotes,decorations.pathmorphing,fit,backgrounds,through,svg.path,topaths,patterns}
\newcommand{\tikzline}[1]{{#1\tikz{\draw[#1,line width=9](0,0)--(0.5,0);}}, }
% 最后一页
\usepackage{lastpage}

% 引用
%\usepackage{hyperref}
\usepackage[unicode]{hyperref}
\hypersetup{colorlinks, bookmarksopen = true,allcolors=black, bookmarksnumbered = true, pdftitle=LaTeX-cn, pdfauthor=K.L Wu, pdfstartview=FitH}

% 奇怪的小定义
\newcommand{\dpar}{\\ \mbox{}}  % 空两行
\newcommand{\qd}[1]{{\bfseries{#1}}}  % 强调
\newcommand{\co}[1]{{\bfseries{#1}}}  % Style of concept
\newcommand{\RED}[1]{{\color{cyan}{#1}}}
\newcommand{\cmmd}[1]{\fbox{\texttt{\char92{}#1}}}
\newcommand{\charef}[1]{第\ref{#1}章}
\newcommand{\secref}[1]{第\ \ref{#1}\ 节}
\newcommand{\pref}[1]{第\pageref{#1}页}
\newcommand{\fref}[1]{图\ref{#1}}
\newcommand{\tref}[1]{表\ref{#1}}

% Quote 环境
\newenvironment{QuoteEnv}[2][]
{\newcommand\Qauthor{#1}\newcommand\Qref{#2}}
{\medskip\begin{flushright}\small ——~\Qauthor\\
		\emph{\Qref}\end{flushright}}
% 字体调用
\newcommand{\myfont}[2]{{\fontfamily{#1}\selectfont #2}}

% 编号列表宏包,并自定义了三个列表
\usepackage[inline]{enumitem}
\setlist[enumerate]{font=\bfseries,itemsep=0pt}
\setlist[itemize]{font=\bfseries,leftmargin=\parindent}
\setlist[description]{font=\bfseries\uline,labelindent=\parindent,itemsep=0pt,parsep=0pt,topsep=0pt,partopsep=0pt}

\newenvironment{fead}{	
	\begin{description}[font=\bfseries\uline,labelindent=\parindent,itemsep=0pt,parsep=0pt,topsep=0pt,partopsep=0pt]}
	{\end{description}}
% 带宽度的
\newenvironment{para}{
	\begin{description}[font=\bfseries\ttfamily,itemsep=0pt,parsep=0pt,topsep=0pt,partopsep=0pt]}
	{\end{description}}
\newenvironment{feae}{
	\begin{enumerate}[font=\bfseries,labelindent=0pt,itemsep=0pt,parsep=0pt,topsep=0pt,partopsep=0pt]}
	{\end{enumerate}}
\newenvironment{feai}{
	\begin{itemize}[font=\bfseries,itemsep=0pt,parsep=0pt,topsep=0pt,partopsep=0pt]}
	{\end{itemize}}
\newenvironment{inlinee}
{\begin{enumerate*}[label=(\arabic*), font=\rmfamily, before=\unskip{:},itemjoin={{;}},itemjoin*={{,以及:}}]}
	{\end{enumerate*}。}

% 目录和章节样式
\usepackage{titlesec}
\usepackage{titletoc}   % 用于目录

\titlecontents{chapter}[1.5em]{}{\contentslabel{1.5em}}{\hspace*{-2em}}{\hfill\contentspage}
\titlecontents{section}[3.3em]{}{\contentslabel{1.8em}}
{\hspace*{-2.3em}}{\titlerule*[8pt]{$\cdot$}\contentspage}
\titlecontents*{subsection}[2.5em]{\small}{\thecontentslabel{} }{}{, \thecontentspage}[;\qquad][.]
% 章节样式
\setcounter{secnumdepth}{3} % 一直到subsubsection
\newcommand{\chaformat}[1]{%
	\parbox[b]{.5\textwidth}{\hfill\bfseries #1}%
	\quad\rule[-12pt]{2pt}{70pt}\quad
	{\fontsize{60}{60}\selectfont\thechapter}}
\titleformat{\chapter}[block]{\hfill\LARGE\sffamily}{}{0pt}{\chaformat}[\vspace{2.5pc}\normalsize
\startcontents\setlength{\lineskiplimit}{0pt}\printcontents{}{1}{\setcounter{tocdepth}{2}\songti}]
\titleformat*{\section}{\centering\Large\bfseries}
\titleformat{\subsubsection}[hang]{\bfseries\large}{\rule{1.5ex}{1.5ex}}{0.5em}{}
% 扩展章节
\newcommand{\starsec}{\noindent\fbox{\S\textit{注意:本章节是一个扩展阅读章节。}}
	\\ \mbox{}}

\renewcommand{\contentsname}{目录}
\renewcommand{\tablename}{表}
\renewcommand\arraystretch{1.2}	% 表格行距
\renewcommand{\figurename}{图}
% 设置不需要浮动体的表格和图像标题
\setlength{\abovecaptionskip}{5pt}
\setlength{\belowcaptionskip}{3pt}
\makeatletter
\newcommand\figcaption{\def\@captype{figure}\caption}
\newcommand\tabcaption{\def\@captype{table}\caption}
\makeatother
% 图表
\usepackage{array,multirow,makecell}
\setlength\extrarowheight{2pt} % 行高增加
\usepackage{diagbox}
\usepackage{longtable}
\usepackage{graphicx,wrapfig}
\graphicspath{{./tikz/}}
\usepackage{animate}
\usepackage{caption,subcaption}
\captionsetup[sub]{labelformat=simple}
\renewcommand{\thesubtable}{(\alph{subtable})}
% 三线表
\usepackage{booktabs}  

% 页面修正宏包
\usepackage{geometry}
\geometry{vmargin = 1in}

% 代码环境
\usepackage{listings}
% 复制代码时不复制行号
\usepackage{accsupp}
\newcommand{\emptyaccsupp}[1]{\BeginAccSupp{ActualText={}}#1\EndAccSupp{}}
\usepackage{tcolorbox}
\tcbuselibrary{listings,skins,breakable,xparse}

% global style
\lstset{
	basicstyle=\small\ttfamily,
	% Word styles
	keywordstyle=\color{blue},
	commentstyle=\color{green!50!black},
	columns=fullflexible,  % Avoid too sparse word spaces
	keepspaces=true,
	% Numbering
	numbers=left,
	numberstyle=\tiny\color{red!75!black}\emptyaccsupp,
	% Lines and Skips
	aboveskip=0pt plus 6pt,
	belowskip=0pt plus 6pt,
	breaklines=true,
	breakatwhitespace=true,
	emptylines=1,  % Avoid >1 consecutive empty lines
	escapeinside=``
}

% TikZ Language Hint
\lstdefinelanguage{tikzlang}{
	sensitive=true,
	morecomment=[n]{[}{]}, % nested comment
	morekeywords={
		draw,clip,filldraw,path,node,coordinate,foreach,pic,
		tikzset
	}
}

% 对于 tcolorbox 中 listings 库的 ''tcblatex'' style 的重现,
% 添加了新的关键词
\lstdefinestyle{latexcn}{
	language=[LaTeX]TeX,
	% More Keywords
	classoffset=0,
	texcsstyle=*\color{blue},
	moretexcs={
		% LaTeX extension
		chapter,section,subsection,setlength,
		thechapter,thesection,thesubsection,theequation,
		chaptermark,chaptername,appendix,
		bibname,refname,bibpreamble,bibfont,citenumfont,bibnumfmt,bibsep,
	},
	classoffset=1,
	texcsstyle=*\color{orange!75!black},
	moretexcs={
		% XeCJK & CTeX
		xeCJKsetup,setCJKmainfont,newCJKfontfamily,CJKfontspec,
		CTEXthechapter,songti,heiti,fangsong,kaishu,yahei,lishu,youyuan,
		% AMSmath / AMSsymb / AMSthm
		middle,text,tag,boldsymbol,mathbb,dddot,ddddot,iint,varoiint,
		dfrac,tfrac,cfrac,leftroot,uproot,underbracket,xleftarrow,xrightarrow,
		overset,underset,sideset,mathring,leqslant,geqslant,because,therefore,
		shortintertext,binom,dbinom,implies,thesubequation,
		impliedby,genfrac,theoremstyle,qedhere,
		% Other math packages
		wideparen,intertext,
		xlongequal,xLeftrightarrow,xleftrightarrow,xLongleftarrow,xLongrightarrow,
		% xcolor
		definecolor,color,textcolor,colorbox,fcolorbox,
		% hyperref
		hyperref,autoref,href,url,nolinkurl,
		% Graph & Table
		includegraphics,graphicspath,scalebox,rotatebox,animategraphics,
		newcolumntype,arraybackslash,multirow,captionsetup,
		thead,multirowcell,makecell,Xhline,Xcline,diagbox,
		toprule,midrule,bottomrule,DeclareFloatingEnvironment,
		% ulem
		uline,uuline,dashuline,dotuline,uwave,sout,xout,
		% fancyhdr
		lhead,chead,rhead,lfoot,cfoot,rfoot,
		fancyhf,fancyhead,fancyfoot,fancypagestyle,
		% fontspec
		newfontfamily,
		% titlesec & titletoc
		titlelabel,titleformat,titlespacing,titleline,titlerule,dottecontents,titlecontents,
		% enumitem
		setlist,
		% Listings & tcolorbox
		lstdefinelanguage,lstdefinestyle,lstset,lstnewenvironment,
		tcbuselibrary,newtcblisting,newtcbox,DeclareTCBListing
		% citation & index: natbib, imakeidx
		setcitestyle,printindex,
		% Other packages
		hologo,lettrine,endfirsthead,endhead,endlastfoot,columncolor,rowcolors,modulolinenumbers,MakeShortVerb,tikz,Hologo
	}
}

% cmd & envi
\newtcbox{\latexline}[1][green]{on line,before upper=\ttfamily\char`\\,
	arc=0pt,outer arc=0pt,colback=#1!10!white,colframe=#1!50!black,
	boxsep=0pt,left=1pt,right=1pt,top=1pt,bottom=1pt,
	boxrule=0pt,bottomrule=1pt,toprule=1pt}
\newtcbox{\envi}[1][violet!70!cyan]{on line,before upper=\ttfamily,
	arc=0pt,outer arc=0pt,colback=#1!10!white,colframe=#1!50!black,
	boxsep=0pt,left=1pt,right=1pt,top=1pt,bottom=1pt,
	boxrule=0pt,bottomrule=1pt,toprule=1pt}
% pkg
\newtcbox{\pkg}[1][orange!70!red]{on line,before upper={\rule[-0.2ex]{0pt}{1ex}\ttfamily},
	arc=0.8ex,colback=#1!30!white,colframe=#1!50!black,
	boxsep=0pt,left=1.5pt,right=1.5pt,top=1pt,bottom=1pt,
	boxrule=1pt}
\newcommand{\tikzkw}[1]{\texttt{#1}}

% tcblisting definitions
\newtcblisting{latex}{breakable,skin=bicolor,colback=gray!30!white,
	colbacklower=white,colframe=cyan!75!black,listing only, 
	left=6mm,top=2pt,bottom=2pt,fontupper=\small,
	listing options={style=latexcn}
}

\NewTCBListing{codeshow}{ !O{listing side text} }{
	skin=bicolor,colback=gray!30!white,
	colbacklower=pink!50!yellow,colframe=cyan!75!black,
	valign lower=center,
	left=6mm,righthand width=0.4\linewidth,fontupper=\small,
	% listing style
	listing options={style=latexcn},#1,
}

% Fix solution from the tcolorbox package author
\makeatletter
\tcbset{
	tikz upper/.style={before upper=\centering\tcb@shield@externalize\begin{tikzpicture}[{#1}],after upper=\end{tikzpicture}},%
	tikz lower/.style={before lower=\centering\tcb@shield@externalize\begin{tikzpicture}[{#1}],after lower=\end{tikzpicture}},%
}
\makeatother

% xparse library required
\NewTCBListing{tikzshow}{ O{} }{
	tikz lower={#1},
	halign lower=center,valign lower=center,
	skin=bicolor,colback=gray!30!white,
	colbacklower=white,colframe=cyan!75!black, 
	left=6mm,righthand width=3.5cm,listing outside text,
	listing options={language=tikzlang}
}

\NewTCBListing{tikzshowenvi}{ O{} }{
	halign lower=center,valign lower=center,
	skin=bicolor,colback=gray!30!white,
	colbacklower=white,colframe=cyan!75!black, 
	left=6mm,righthand width=3.5cm,listing outside text,
	listing options={language=tikzlang},#1
}
% inline tikz draw
%\newcommand{\tikzline}{def}

% 附录
% \usepackage{appendix}
\renewcommand{\appendixname}{App.}

% 行号
\usepackage{lineno}

% 索引与参考文献
\usepackage{imakeidx}
\newcommand{\tikzidx}[1]{\index{\char`\\ #1}}
\newcommand{\tikzoptstyle}[1]{\texttt{#1}}
\newcommand{\tikzopt}[2][draw]{\tikzoptstyle{#2}\index{\char`\\ #1!#2}\ }
\renewcommand{\indexname}{\tikzz 命令索引}
\makeindex[intoc]

\bibliographystyle{plain}
\renewcommand{\bibname}{参考文献}
\usepackage[numindex,numbib]{tocbibind}
\usepackage[square,super,sort&compress]{natbib}
\begin{document}


\title{简单粗暴\LaTeX\ }
\author{K.L Wu\\
	{\kaishu 本手册是\href{https://github.com/wklchris/Note-by-LaTeX}{wklchris-GitHub}的\LaTeX{}-cn项目}
}
\date{当前版本号:v1.6.4-pre\\
	最后更新于:\today}

\maketitle

\setlength{\lineskiplimit}{0pt}
\tableofcontents
\setlength{\lineskiplimit}{3pt}

\chapter{序}

\noindent{\Huge\xinwei 第一稿序}\dpar\dpar

其实在之前我是有一稿手册的,开始撰写的日期大概在2015年4月.但是自己觉得写得太烂,因此索性推倒重写了这一版.这一版的主要特征是:
\begin{feae}
  \item 我希望能够吸引初学者快速上手,解决手头的问题.因此去掉了枯燥的讲解和无穷无尽的宏包用法介绍,直接使用实例;
  \item 力求突出实用性.当然,也会提点一些可以深入学习的内容,读者可以自行查阅,或者阅读本手册中的扩展阅读章节(即带星号*的章节).
  \item 本手册使用的编辑器为\TeX\ Studio,而非之前的商业软件WinEdt. 这使得学习\LaTeX\ 的门槛更低.当然了,你有权使用任何编辑器.
\end{feae}

手册主体分为六大部分\cite{LHY2013latex,lamport1994latex,frank2004latex,partl2016,Casteleyn2016tikz,tikzmanual}:
\begin{fead}
\item[写给读者*] 介绍\LaTeX\ 背景、优缺点、适用情形.
\item[基础] 包括标点、缩进、距离、章节、字体、颜色、注释、引用、封面、目录、列表、图表、页面等详细内容.
\item[数学排版] 包括数学符号、公式、编号等内容.
\item[进阶] 主要是自定义命令,帮助你更高效、更简洁地书写你的文档.
\item[\tikzz\  绘图*] 附加章节,需要读者取消注释后重编译.
\item[附录] 帮助你快速查找一些你想要的东西.
\end{fead}

由于工作全部由我一人完成,限于视野,难免存在错漏之处.恳请读者指正.如遇到的手册中无法解决的问题,欢迎向我提出.推荐书目可参考本手册附录.

最后,还要感谢在\LaTeX\ 学习中为我解答疑惑的同学,特别是来自\LaTeX\ 度吧的吧友;本手册中许多的解决方案都是由他们提供的.我谨在此记录.

\vfill

\begin{flushright}
Mail: wklchris@hotmail.com\dpar

Chris Wu

September 17, 2016 于Davis, CA
\end{flushright}
\clearpage
\noindent{\Huge\xinwei 更新日志:}\dpar\dpar

版本号以 $x.y.z$ 的形式公示.当$z=0$时,为决定正式 release 的版本.

v1.6.0 更新 --- 2017年6月15日:

v1.6.1 更新 --- 2017年8月9日;

v1.6.2 更新 --- 2017年10月5-15日.

v1.6.3 更新 --- 2018年3月22日.\dpar

以下内容中,未以括号注明的项是在v1.6.0中更新的.

\begin{feai}
\item 重要更新:
  \begin{feai}
  \item \textbf{\sout{\tikzz\  相关的内容停止更新}.原因是现有的其他软件绘图功能强大,导出为 pdf 格式的矢量图后也易于调用;\tikzz\  相比之下学习成本过高.}\textcolor{red}{\textbf{笔者又决定重启\tikzz\ 章节}}.(v1.6.3)
  \item \textbf{重新添加了 \tikzz\  章节} —— 不过仍是之前弃笔时的版本,预计将于 v1.7 更新.
  \item \textbf{增加了加快\xelatex 调用字体速度的方式}.参考“中文支持与CJK字体”一节\texttt{fc-cache}相关内容.(v1.6.1)
  \end{feai}
\item 字体更换:思源宋体.
\item 编辑了 Head.tex 文件,使之更易阅读.
\item [添加]宏包\pkg{animate}:在 PDF 中展示动态图.(v1.6.1)
\item [添加]宏包\pkg{tocbibind}:
  \begin{feai}
  \item 将目录本身编入目录项.
  \item 将参考文献章节编号、编入目录项.
  \item 将索引章节编号、编入目录项.
  \end{feai}
\item [更新]宏包\pkg{dcolumn},更详细说明了如何在表格中使用小数点对齐.(v1.6.2)
\item [更新]宏包\pkg{fancyvbr}:更详细说明了抄录环境\envi{BVerbatim},如何提供居中支持.(v1.6.2)
\item [更新]宏包\pkg{xeCJK}:
  \begin{feai}
  \item 参数 CJKspace.该功能在新版宏包中已修复.
  \item 命令 \latexline{setCJKmainfont},可指定字体文件名.
  \end{feai}
\item 数学内容:
  \begin{feai}
  \item [添加]宏包\pkg{extarrows}:长等号命令.
  \item [添加]命令\latexline{numberthis},用于给多行公式中的某行编号.(v1.6.2)
  \item [添加]命令\latexline{allowdisplaybreaks},用于支持多行公式环境换页.
  \end{feai}
\item 拆分了文档,并修改了“文档拆分”一节的内容.(v1.6.3)
\item 修复了一些错别字与无效的文档内跳转链接.
\end{feai}

\mbox{}

更早版本的更新细节,请到\href{https://github.com/wklchris/Note-by-LaTeX/releases}{Project Release Webpage}浏览.

% Main Contents

\include{chapters/To-Readers}
\include{chapters/LaTeX-Basics}
\include{chapters/Play-with-Math}
\include{chapters/LaTeX-Advanced-Skills}
% \include{chapters/TikZ}

% Appendices

% 参考文献
\bibliography{Bib}

% 附录
\clearpage
\appendix
% 重定义附录的chapter样式
\renewcommand{\chaformat}[1]{%
	\parbox[b]{.5\textwidth}{\raggedleft\bfseries \S 附录 \\ \vspace{0.2ex} #1} \quad\rule[-12pt]{2pt}{70pt}\quad
	{\fontsize{60}{60}\selectfont\thechapter}}

\chapter{注音符号}
\label{app:phonetic}
% 这里不用>{\ttfamily}而用\verb是为了减少报错可能
\begin{center}
\tabcaption{注音符号与特殊符号}
\begin{tabular}{|*{4}{>{\centering}p{3em} @{-\hspace{1em}} p{3em}|}}
\hline
\texttt{样式} & 命令 & \texttt{样式} & 命令 & \texttt{样式} & 命令 & \texttt{样式} & 命令 \\
\hline
\=o  & \verb|\=o|  & \'o  & \verb|\'o|  & \v o & \verb|\v o|  & \`o   & \verb|\`o|  \\
\^o  & \verb|\^o|  & \"o  & \verb|\"o|  & \.o  & \verb|\.o|   & \H o  & \verb|\H{o}| \\
\d o & \verb|\d{o}| & \u o & \verb|\u{o}| & \b o & \verb|\b{o}|  & \t oo & \verb|\t{oo}|\\
\multicolumn{2}{|c@{\bf --}}{$\tilde{o}$} & \multicolumn{2}{@{\bf --}c|}{\tt{\$$\backslash$tilde\{o\}\$}} &%
\multicolumn{2}{c@{\bf --}}{$\hat{o}$}    & \multicolumn{2}{@{\bf --}c|}{\tt{\$$\backslash$hat\{o\}\$}}\\
\multicolumn{8}{|c|}{} \\
\o  & \verb|\o|  & \O  & \verb|\O|  & \i  & \verb|\i|  & \j  & \verb|\j| \\
\aa & \verb|\aa| & \AA & \verb|\AA| & \ae & \verb|\ae| & \AE & \verb|\AE|\\
\oe & \verb|\oe| & \OE & \verb|\OE| & !`  & \verb|!`|  & ?`  & \verb|?`| \\
\hline
\end{tabular}
\end{center}

\mbox{}

\begin{center}
\tabcaption{国际音标输入表(部分)}
\begin{tabular}{|*{3}{>{\rmfamily}c !{-} >{\ttfamily}p{7.5em}|}}
\hline
\texttt{样式} & 命令 & \texttt{样式} & 命令 & \texttt{样式} & 命令 \\
\hline
\textdzlig & \char92textdzlig & \textesh & \char92textesh & \textteshlig & \char92textteshlig \\
\textdyoghlig & \char92textdyoghlig & \textturnv & \char92textturnv & \textschwa & \char92textschwa \\
\textscriptg & \char92textscriptg & \texttheta & \char92texttheta & \textupsilon & \char92textupsilon \\
\textscripta & \char92textscripta & \dh & \char92dh & \textepsilon & \char92textepsilon \\
\textopeno & \char92textopeno & \textyogh & \char92textyogh & \ng & \char92ng \\
\hline
\multicolumn{2}{|c|}{重音} & \multicolumn{2}{c|}{次重音} & \multicolumn{2}{c|}{长音节} \\
\textprimstress & \char92{}textprimstress & \textsecstress & \char92textsecstress & \textlengthmark & \char92textlengthmark \\
\hline
\end{tabular}
\end{center}

\textit{注:\texttt{\char92dh}命令在非CJK文档中有时编译会出现问题}.

\chapter{建议与其他}

除了参考文献列表中给出的书籍以外,我还推荐你用控制台在\TeX{} Live中能找到的以下书籍:

\medskip\begin{para}
\item[texdoc usrguide] \TeX\ Live自带的用户手册.
\item[texdoc clsguide] \TeX\ Live自带的文档类和宏包编写手册.
\item[texdoc fntguide] \TeX\ Live自带的字体使用手册.
\item[texdoc symbols-a4] 一份速查表,基本上所有的\LaTeX\ 字符命令都在这里了.
\item[texdoc latexcheat] 很有趣的命令表,只有两页.
\item[texdoc impatient] \emph{\TeX{} for the Impatient}, 一本介绍底层\TeX\ 的书.这也是我阅读的第一本\TeX\ 书,Knuth的\emph{The \TeX\ book}虽然血统正但是难啃啊.本书中译本在:\url{https://bitbucket.org/zohooo/impatient/wiki/Home}
\item[texdoc texbytopic] \emph{\TeX{} by Topic}, 个人觉得不如上面那本,但也许只是叙述方式不一样吧.
\end{para}
\bigskip

\noindent 你可能还需要的功能:
\begin{description}
\item[\pkg{mhchem}] 该宏包用于输入化学式,提供了\latexline{ce}命令.
\end{description}

% \tikzz\ 索引
%\setlength{\columnseprule}{0pt}
%\printindex

\end{document}
